%% -*- coding: utf-8 -*-
\documentclass[12pt,a4paper]{scrartcl} 
\usepackage[utf8]{inputenc}
\usepackage[english,russian]{babel}
\usepackage{indentfirst}
\usepackage{misccorr}
\usepackage{graphicx}
\usepackage{amsmath}
\begin{document}
	\begin{titlepage}
		\begin{center}
			\large
			МИНИСТЕРСТВО НАУКИ И ВЫСШЕГО ОБРАЗОВАНИЯ РОССИЙСКОЙ ФЕДЕРАЦИИ
			
			Федеральное государственное бюджетное образовательное учреждение высшего образования
			
			\textbf{АДЫГЕЙСКИЙ ГОСУДАРСТВЕННЫЙ УНИВЕРСИТЕТ}
			\vspace{0.25cm}
			
			Инженерно-физический факультет
			
			Кафедра автоматизированных систем обработки информации и управления
			\vfill
			
			\vfill
			
			\textsc{Отчет по практике}\\[5mm]
			
			{\LARGE Программаная реализация Хеш-таблицы \textit}
			\bigskip
			
			2 курс, группа 2УТС
		\end{center}
		\vfill
		
		\newlength{\ML}
		\settowidth{\ML}{«\underline{\hspace{0.7cm}}» \underline{\hspace{2cm}}}
		\hfill\begin{minipage}{0.5\textwidth}
			Выполнил:\\
			\underline{\hspace{\ML}} А.\,А.~Ашла\\
			«\underline{\hspace{0.7cm}}» \underline{\hspace{2cm}} 2023 г.
		\end{minipage}%
		\bigskip
		
		\hfill\begin{minipage}{0.5\textwidth}
			Руководитель:\\
			\underline{\hspace{\ML}} С.\,В.~Теплоухов\\
			«\underline{\hspace{0.7cm}}» \underline{\hspace{2cm}} 2023 г.
		\end{minipage}%
		\vfill
		
		\begin{center}
			Майкоп, 2023 г.
		\end{center}
	\end{titlepage}
	
	\section{Введение}
	\label{sec:intro}
	
	% Что должно быть во введении
	\begin{enumerate}
		\item Текстовая формулировка задачи
		\item Пример кода, решающего данную задачу
		\item График
		\item Скриншот программы
	\end{enumerate}
	
	Пример приведен в пункте ~\ref{sec:exp} на стр.~\pageref{sec:exp}.
	
	\section{Ход работы}
	\label{sec:exp}
	
	\subsection{Код приложения}
	\label{sec:exp:code}
	\begin{verbatim}
		public static void main(String[] args) {
			HashTable hashTable = new HashTable(8);
			
			//ВСТАВКА ЭЛЕМЕНТОВ
			hashTable.insert(11);
			hashTable.insert(5);
			hashTable.insert(15);
			hashTable.insert(20);
			hashTable.insert(2);
			
			//ДЕБАЖИМ ТЕКУЩУЙ ХЭШ
			hashTable.printTable();
			
			//ПРОВЕРЯЕМ НАЛИЧИЕ
			System.out.println("Search 15: " + hashTable.search(15));
			System.out.println("Search 7: " + hashTable.search(7));
			
			//УДАЛЯЕМ ЭЛЕМЕНТЫ
			hashTable.delete(15);
			hashTable.delete(7);
			
			//ДЕБАЖИМ ТЕКУЩУЙ ХЭШ
			hashTable.printTable();
		}
		
		public class HashTable {
			
			//РАЗМЕР ТАБЛИЦЫ
			private final int size;
			//СПИСКИ
			private final LinkedList<Integer>[] table;
			
			public HashTable(int size) {
				this.size = size;
				this.table = new LinkedList[size];
				//ИНИЦИЛИЗИРУЕМ КАЖДЫЙ ЭЛЕМЕНТ МАССИВА
				Arrays.fill(table, new LinkedList<>());
			}
			
			/* ПОЛУЧАЕМ ИНДЕКС В МАССИВЕ */
			private int hash(int key) {
				return key % size;
			}
			
			/* ВСТАВКА ЭЛЕМЕНТА */
			public void insert(int key) {
				//ПОЛУЧАЕМ ИНДЕКС В МАССИВЕ
				int index = hash(key);
				//СПИСОК ПОД ЭТОТ ИНДЕКТ
				LinkedList<Integer> list = table[index];
				if(!list.contains(key)) {
					list.add(key);
				}
			}
			
			/* ПОИСК В ТАБЛИЦЕ */
			public boolean search(int key) {
				int index = hash(key);
				LinkedList<Integer> list = table[index];
				return list.contains(key);
			}
			
			/* УДАЛЕНИЕ ЭЛЕМЕНТА В ТАБЛИЦЕ */
			public void delete(int key) {
				int index = hash(key);
				LinkedList<Integer> list = table[index];
				//УДАЛЯЕМ ПЕРВЫЙ ЭЛЕМЕНТ В КОЛЛЕКЦИИ
				list.removeFirstOccurrence(key);
			}
			
			public void printTable() {
				for(int i = 0; i < size; i++) {
					LinkedList<Integer> list = table[i];
					System.out.print("Index " + i + ": ");
					for(int number : list) {
						System.out.print(number + " ");
					}
					System.out.println();
				}
			}
		}
		
	\end{verbatim}
\end{document}